\section{Database design}\label{Section label}
A high-level, "conceptual" ERD describing your database using Chen notation.
\subsection{Subsection}\label{subsection}
The Database Design (groupwork - 10\%) \newline
Create an Entity-Relationship Diagram of your project database using the Chen model
diagram. Your database design should include (Minimum requirements):
\begin{itemize}
    \item at least 8 interrelated entity sets, and at least 40 attributes (total),
    \item enough relationships to connect each entity set to at least one other entity set.
    \item be sure to indicate identifying attributes (primary keys)
    \item put relationship constraints (i.e., connectivity, cardinality and participation).
    \item one or more M:N relationships,
    \item one or more recursive relationship
    \item one or more multivalued attributes.
\end{itemize}

When you are confident you have a fully qualified diagram, it is desirable that you render your diagram using an E/R design tool, such as \newline(http://www.conceptdraw.com/en/products/cd5/main.php) (this gives you a free trial for 30 days) or http://drive.draw.io/ (There is an E/R style diagram under the "Software design"templates). Alternatively, you can draw your diagram using a simple Microsoft word diagramming tools and shapes.

It is very important to discuss your ER design with your tutors. During week 10 of the semester each team will submit a document detailing the domain of interest, the database analysis and your ER design description. Students will receive formative feedback about the briefing report within a week.