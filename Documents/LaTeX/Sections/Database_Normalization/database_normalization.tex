\section{Database normalization}\label{Section label}
A low-level, "logical" ERM. A relational schema for your database, with primary and foreign keys specified appropriately. Add comments as necessary to make it clear which are the special tables such as bridging tables and tables introduced to handle multivalued attributes. (This is to help the reader - to save them having to work it out from your conceptual and logical ERDs.). A list of the functional dependencies for your scheme and collection of normalized tables obtained through a normalization process.
\subsection{Subsection}\label{subsection}
The Database Normalization (groupwork - 10\%) \newline
Convert your ERD diagram to a relational database schema, i.e. a set of tables, each with appropriate attributes, a primary key, and appropriate foreign keys. A schema diagram for your database, with primary and foreign keys specified appropriately. This is one where every entity type in the diagram corresponds to a table. In this ERM you must have introduced bridging types, must have introduced the special implementation of the symmetric relationship, and must have introduced a handling of the multivalued attributes. Go through the normalization process to come up with a collection of tables that are in third normal forms. Primary and foreign keys should be specified appropriately.